\begin{flushleft}
    % Ubah kalimat berikut sesuai dengan nama departemen dan fakultas
    \textbf{Departemen Teknik Komputer - FTEIC}\\
    \textbf{Institut Teknologi Sepuluh Nopember}\\
  \end{flushleft}
  
  \begin{center}
    % Ubah detail mata kuliah berikut sesuai dengan yang ditentukan oleh departemen
    \underline{\textbf{EC184701 - PRA TUGAS AKHIR (2 SKS)}}
  \end{center}
  
  \begin{adjustwidth}{-0.2cm}{}
    \begin{tabular}{lcp{0.7\linewidth}}
  
      % Ubah kalimat-kalimat berikut sesuai dengan nama dan NRP mahasiswa
      Nama Mahasiswa &:& Alan Luthfi \\
      Nomor Pokok &:& 07211840000063 \\
  
      % Ubah kalimat berikut sesuai dengan semester pengajuan proposal
      Semester &:& Ganjil 2021/2022 \\
  
      % Ubah kalimat-kalimat berikut sesuai dengan nama-nama dosen pembimbing
      Dosen Pembimbing &:& 1. Dr. Eko Mulyanto Yuniarno ST., MT \\
      & & 2. \\
  
      % Ubah kalimat berikut sesuai dengan judul tugas akhir
      Judul Tugas Akhir &:& \textbf{Menghitung Luas Bangun Datar pada Papan Tulis Menggunakan YOLO} \\
      %& & \textbf{Covolutional Neural Network Berbasis Citra Wajah} \\
  
      Uraian Tugas Akhir &:& \\
    \end{tabular}
  \end{adjustwidth}
  
  % Ubah paragraf berikut sesuai dengan uraian dari tugas akhir
  Papan tulis pintar memiliki potensial untuk menjadi alat pembelajaran revolusioner kedua setelah papan tulis hitam tradisional, karena papan tulis pintar yang bisa disematkan dalam ruang kelas modern bisa menggerakan sekolah ke arah mode operasi digital yang lebih terintegrasi. Pada papan tulis pintar harus memiliki fitur yang dapat membedakan papan tulis pintar dengan papan tulis biasa, karena papan tulis pintar memiliki fitur-fitur atau kegunaan lebih superior daripada papan tulis biasa. Oleh karena itu diperlukan pengembangan fitur pada papan tulis pintar. Tujuan penelitian adaah membuat program yang dapat mendeteksi bangun datar dan parameternya lalu menghitung luas bangun datar pada papan tulis pintar. Metode yang akan digunakan adalah dengan menggunakan YOLO sebagai framework pengerjaan dalam pembuatan program deteksi bangun datar dan parameternya.
  \vspace{1ex}
  
  \begin{flushright}
    % Ubah kalimat berikut sesuai dengan tempat, bulan, dan tahun penulisan
    Surabaya, Desember 2021
  \end{flushright}
  \vspace{1ex}
  
  \begin{center}
  
    \begin{multicols}{2}
  
      Dosen Pembimbing 1
      \vspace{12ex}
  
      % Ubah kalimat-kalimat berikut sesuai dengan nama dan NIP dosen pembimbing pertama
      \underline{[Dr. Eko Mulyanto Yuniarno, S.T., M.T.]} \\
      NIP. 196806011995121000
  
      \columnbreak
  
      Dosen Pembimbing 2
      \vspace{12ex}
  
      % Ubah kalimat-kalimat berikut sesuai dengan nama dan NIP dosen pembimbing kedua
      \underline{} \\
      NIP. 
  
    \end{multicols}
    \vspace{6ex}
  
    Mengetahui, \\
    % Ubah kalimat berikut sesuai dengan jabatan kepala departemen
    Kepala Departemen Teknik Komputer FTEIC - ITS
    \vspace{12ex}
  
    % Ubah kalimat-kalimat berikut sesuai dengan nama dan NIP kepala departemen
    \underline{Dr. Supeno Mardi Susiki Nugroho, S.T., M.T.} \\
    NIP. 197003131995121001
  
  \end{center}