\section{PENDAHULUAN}

\subsection{Latar Belakang} 

% Ubah paragraf-paragraf berikut sesuai dengan latar belakang dari tugas akhir
Alat pengajaran revolusioner pertama yaitu papan tulis hitam digunakan pada pengajaran dalam ruang kelas pada tahun 1801 dan memiliki dampak yang besar dalam pengajaran selama 200 tahun kedepan. Papan tulis pintar memiliki potensial untuk menjadi alat pengajaran revolusioner kedua. Seperti halnya papan tulis hitam yang menjadi bagian dari kunci ruang kelas pada abad sembilan belas dan abad dua puluh, papan tulis pintar memiliki kapabilitas untuk menjadi bagian dari kunci ruang kelas digital pada abad dua puluh satu. Meskipun relatif baru, papan tulis pintar memiliki kapasitas yang sama untuk merubah fundamental dan merevolusionerkan cara mengajar.
Dalam hal yang sama pada papan tulis hitam di zaman lampau yang merupakan teknologi yang digunakan oleh sekolah tradisional, papan tulis pintar sudah menampakkan fasilitas yang bisa digunakan oleh sekolah digital. Karena kapasitas papan tulis pintar yang bisa disematkan dalam ruang kelas modern, papan tulis pintar bisa menjadi katalis yang menggerakkan sekolah dari model tradisional berbasis kertas ke arah mode operasi digital yang lebih terintegrasi. Model sekolah tradisional berbasis kertas sudah ada dalam waktu yang cukup lama, namun kita mulai melihat pergantian pada sekolah diseluruh dunia untuk memaksimalkan potensial pembelajaran digital dan memanfaatkan keuntungan daripada kesempatan evolusi edukasi yang dibawa oleh dunia digital.
Namun perlu diingat bahwa ini adalah permulaan daripada revolusi. Tantangan yang dihadapi oleh guru dalam pengembangan pada ruang kelas digital adalah untuk melihat potensial yang tersedia lalu memanfaatkannya, dan berkolaborasi dengan rekan kerja maupun peserta didik untuk menggunakan alat pembelajaran dalam dunia digital secara efektif.  \citep{Lant2016}



\subsection{Permasalahan}

% Ubah paragraf berikut sesuai dengan permasalahan dari tugas akhir
Pada papan tulis pintar harus memiliki fitur yang dapat membedakan papan tulis pintar dengan papan tulis biasa, karena papan tulis pintar memiliki fitur-fitur atau kegunaan lebih superior daripada papan tulis biasa. Oleh karena itu diperlukan pengembangan fitur pada papan tulis pintar. 


\subsection{Tujuan Penelitian}

% Ubah paragraf berikut sesuai dengan tujuan penelitian dari tugas akhir
Membuat program yang dapat mendeteksi bangun datar dan parameternya lalu menghitung luas bangun datar pada papan tulis menggunakan YOLO.