\section{TINJAUAN PUSTAKA}

\subsection{Deep Learning}

% Contoh penggunaan referensi dari pustaka
% Newton pernah merumuskan \citep{Newton1687} bahwa \lipsum[8]
% Contoh penggunaan referensi dari persamaan
% Kemudian menjadi persamaan seperti pada persamaan \ref{eq:FirstLaw}.

Deep learning memungkinkan model komputasi yang terdiri dari beberapa lapisan pemrosesan untuk mempelajari representasi data dengan berbagai tingkat abstraksi. Metode-metode ini telah secara dramatis meningkatkan state-of-the-art dalam pengenalan suara, pengenalan objek visual, deteksi objek dan banyak domain lainnya seperti penemuan obat dan genomik. Deep learning menemukan struktur rumit dalam kumpulan data besar dengan menggunakan algoritma backpropagation untuk menunjukkan bagaimana mesin harus mengubah parameter internalnya yang digunakan untuk menghitung representasi di setiap lapisan dari representasi di lapisan sebelumnya. Deep convolutional nets telah menghasilkan terobosan dalam pemrosesan gambar, video, ucapan, dan audio, sedangkan jaring berulang telah menyoroti data berurutan seperti teks dan ucapan.  \citep{article}

\subsection{YOLO}
YOLO (You Only Look Once) merupakan sistem deteksi objek secara waktu nyata. YOLO merupakan single CNN (Convulotional Neural Network) yang secara bersamaan memprediksi lebih dari satu bounding boxes dan kelas pada satu gambar dalam satu kali pindai. Framework ini dikembangkan oleh Redmon J., Divvala S., Girshick R., Farhadi A. arsitektur jaringannya terinspirasi dari model GoogLeNet untuk klasifikasi gambar. jaringan YOLO memiliki 24 convolutional layer diikuti dengan dua layer yang terhubung.
Saat ini, ada tiga versi YOLO yaitu YOLOv1, YOLOv2, dan YOLOv3. YOLOv2 merupakan versi yang telah dikembangkan dari YOLOv1 yang mana tetap memiliki kecepatan yang sama namun dengan penambahan batch normalization, anchor boxes dan high-resolution classifier. pada YOLOv3, fitur ektraksi yang lebih baik diperkenalkan dilanjutkan dengan perkenalan 53 convolutional layer terlatih pada ImageNet. Tingkat ketelitian YOLOv3 lebih baik dari YOLOv2 namun lebih lambat karena lebih banyak layer. \citep{Redmon_2016_CVPR}

% % input gambar
% \begin{figure} [H] \centering
%     % Nama dari file gambar yang diinputkan
%     \includegraphics[scale=0.6]{gambar/umur.png}
%     % Keterangan gambar yang diinputkan
%     \caption{Kategori umur menurut Depkes. RI (2009)}
%     % Label referensi dari gambar yang diinputkan
%     \label{fig:Umur}
% \end{figure}

\subsection{Bangun Datar}
Dalam geometri, bentuk 2D didefinisikan sebagai bangun datar yang hanya memiliki dua dimensi yaitu panjang dan lebar. Mereka tidak memiliki ketebalan apapun dan hanya dapat diukur dengan dua dimensi. Lingkaran, persegi, persegi panjang, dan segitiga adalah beberapa contoh benda dua dimensi dan bentuk-bentuk ini dapat digambar di atas kertas. Semua bentuk 2D memiliki sisi, simpul (sudut), dan sudut internal, kecuali lingkaran, yang merupakan sosok melengkung. Bentuk 2D dengan setidaknya tiga sisi lurus disebut poligon dan itu termasuk segitiga, bujur sangkar, dan segi empat.

% \subsection{Etnis}
% Kata etnis mengacu pada suatu golongan atau kelompok manusia yang anggota - anggotanya mengidentifikasikan 
% dirinya dengan sesamanya, biasanya berdasarkan garis keturunan dan adat yang dianggap sama. Identitas 
% etnis ditandai oleh pengakuan dari orang lain akan ciri khas kelompok tersebut seperti kesamaan budaya, 
% bahasa, agama, perilaku, dan ciri-ciri biologis.
% % input gambar
% \begin{figure} [H] \centering
%     % Nama dari file gambar yang diinputkan
%     \includegraphics[scale=0.6]{gambar/etnik.png}
%     % Keterangan gambar yang diinputkan
%     \caption{Macam-macam Etnik di dunia}
%     % Label referensi dari gambar yang diinputkan
%     \label{fig:Etnik}
% \end{figure}

% \subsection{Visi Komputer}
% Visi komputer adalah bidang ilmiah interdisipliner yang membahas bagaimana komputer dapat memperoleh 
% pemahaman tingkat tinggi dari gambar atau video digital. Dari perspektif teknik, bidang ini berupaya 
% mengotomatiskan tugas-tugas yang dapat dilakukan oleh sistem pengelihatan  manusia. Tugas  penglihatan 
% komputer   meliputi metode untuk memperoleh, memproses, menganalisis dan memahami gambar digital, dan 
% ekstraksi data dimensi tinggi dari dunia nyata untuk menghasilkan informasi numerik atau simbolis, 
% misalnya dalam bentuk keputusan. Pengertian dalam konteks ini berarti transformasi gambar visual 
% (input retina) menjadi deskripsi mengenai dunia sekitar yang dapat berinteraksi dengan proses pemikiran 
% lain dan memperoleh tindakan yang sesuai. Pemahaman gambar ini dapat dilihat sebagai penguraian informasi 
% simbolik dari data gambar menggunakan model yang dibangun dengan bantuan geometri, fisika, statistik, dan 
% teori pembelajaran. Sub-domain dari pengelihatan komputer meliputi rekonstruksi adegan, deteksi peristiwa, 
% pelacakan video, pengenalan objek, estimasi pose 3D, pembelajaran, pengindeksan, estimasi gerakan, dan 
% pemulihan gambar[9].

% \subsection{Machine Learning}
% Machine Learning (ML) atau Pembelajaran Mesin merupakan bagian dari Artificial Intelligence (AI) yang 
% bertujuan untuk memberi optimalisasi dalam kriteria dengan cara menganalisa sampel data yang terdahulu 
% yang sudah disimpan atau direkam untuk menghasilkan sebuah prediksi. Sehingga manusia tidak perlu 
% mengindentifikasi sebuah proses sepenuhnya, karena dengan Machine Learning, komputer mampu membuat pola 
% untuk membuat keputusan. Machine Learning melakukan training yang merupakan proses pembelajaran terhadap 
% model data yang sudah terdefinisikan ke beberapa parameter (data training) yang menghasilkan beberapa 
% pola sehingga komputer dapat melakukan proses klasifikasi berdasarkan pola atau ciri-ciri yang sudah 
% didapatkan dalam proses training. Kemudian komputer dapat memberikan sebuah prediksi pada data baru 
% selanjutnya berdasarkan hasil training. Machine Learning dapat memberi solusi dalam berbagai permasalahan
%  seperti Computer Vision (Visi Komputer), Speech Recognition (Pengenalan Suara) dan Robotics (Robotika)[10].

% \subsection{Deep Learning}
%  Deep Learning merupakan artificial neural network yang memiliki banyak layer dan synapse weight. 
%  Deep learning dapat menemukan relasi tersembunyi atau pola yang rumit antara input dan output, yang 
%  tidak dapat diselesaikan menggunakan multilayer perceptron. Keuntungan  utama  deep  learning  yaitu 
%  mampu merubah data dari nolinearly separable menjadi linearly separable melalui serangkaian transformasi 
%  (hidden layers). Selain itu, deep learning juga mampu mencari decision boundary yang berbentuk non-linier
%  , serta mengsimulasikan interaksi non-linier antar fitur. Jadi, input ditransformasikan secara 
%  non-linier sampai akhirnya pada output, berbentuk distribusi class-assignment[11].

%  \begin{figure} [H] \centering
%     % Nama dari file gambar yang diinputkan
%     \includegraphics[scale=0.6]{gambar/deeplearning.png}
%     % Keterangan gambar yang diinputkan
%     \caption{Deep Learning 4 layer}
%     % Label referensi dari gambar yang diinputkan
%     \label{fig:Deep Learning}
% \end{figure}

% \subsection{Convolutional Neural Network (CNN)}
% Convolutional Neural Network (CNN) merupakan cabang dari Multilayer Perceptron (MLP) yang digunakan untuk
% mengolah data dua dimensi. CNN memiliki kedalaman jaringan yang tinggi sehingga CNN termasuk dalam jenis
% Deep Neural Network. Perbedaan CNN dengan MLP terdapat pada neuron dimana pada MLP setiap neuron hanya
% berukuran satu dimensi, sedangkan CNN setiap neuronnya berukuran dua dimensi. Pada CNN, operasi linier
% menggunakan operasi konvolusi[12].

% \subsection{Image Processing}
% Image Processing atau Pengolahan Citra merupakan teknik dalam pemrosesan gambar dengan input berupa 
% citra dua dimensi yang bertujuan untuk menyempurnakan citra atau mendapatkan informasi yang berguna 
% untuk diolah menjadi beberapa keputusan. Dalam operasi pemrosesan citra, operasi yang sering dilakukan 
% dalam format gambar grayscale. Gambar grayscale didapatkan dari pemrosesan gambar berwarna yang 
% didekomposisi menjadi komponen merah (R), hijau (G) dan biru (B) yang diproses secara independen sebagai 
% gambar grayscale. Image Processing terbagi menjadi dalam tiga tingkatan[13]:
%     \begin{enumerate}
%         \item Low-Level Image Processing \\
%         Low-Level Image Processing merupakan operasi sederhana dalam pengolahan gambar dimana input dan 
%         output berupa gambar. Contoh: contrast enchancement dan noise reduction.
%         \item Mid-Level Image Processing \\
%         Mid-Level Image Processing merupakan operasi pengolahan gambar yang melibatkan ekstrasi atribut dari 
%         gambar input. Contoh: edges, contours dan regions.
%         \item High-Level Image Processing \\
%         High-Level Image Processing merupakan merupakan kategoriyang melibatkan pemrosesan gambar kompleks 
%         yang terkait dengan analisis dan interpretasi konten dalam sebuah keadaan untuk pengambilan keputusan.
%     \end{enumerate}

% \subsection{Digital Image}
% Digital Image merupakan fungsi dua dimensi f(x,y) yang merupakan proyeksi dari bentuk tiga dimensi kedalam 
% bentuk dua dimensi dimana x dan y merupakan lokasi elemen gambar atau piksel yang berisikan nilai. Ketika
% nilai x,y dan intensitasnya berupa diskrit, maka gambar tersebut dapat dikategorikan sebagai digital
% image. Secara matematis, digital image adalah representasi matriks dari gambar dua dimensi menggunakan
% piksel. Setiap piksel  diwakili  oleh  nilai  numerik. Untuk  gambar  grayscale,  hanya  memiliki  
% satu  nilai dengan kisaran antara 0-255.Pada Gambar 2.5, untuk gambar yang berwarna, memiliki tiga 
% nilai yang mewakili merah (R), hijau (G) dan biru (B) yang masing-masing memiliki kisaran nilai yang 
% sama antara 0-255. Jika suatu gambar hanya memiliki dua intensitas, gambar tersebut dikenal sebagai 
% binary image[13].
