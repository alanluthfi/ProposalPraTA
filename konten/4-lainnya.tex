\section{HASIL YANG DIHARAPKAN}

Hasil yang diharapkan dalam tugas akhir ini adalah berhasil membuat program yang dapat mendeteksi bangun datar dan parameternya lalu menghitung luas bangun datar pada papan tulis dengan tingkat ketelitian yang baik.

\section{RENCANA KERJA}

% Ubah tabel berikut sesuai dengan isi dari rencana kerja
\newcommand{\w}{}
\newcommand{\G}{\cellcolor{gray}}
\begin{table}[h!]
  \begin{tabular}{|p{3.5cm}|c|c|c|c|c|c|c|c|c|c|c|c|c|c|c|c|}

    \hline
    \multirow{2}{*}{Kegiatan} & \multicolumn{16}{|c|}{Minggu} \\
    \cline{2-17} &
    1 & 2 & 3 & 4 & 5 & 6 & 7 & 8 & 9 & 10 & 11 & 12 & 13 & 14 & 15 & 16 \\
    \hline

    % Gunakan \G untuk mengisi sel dan \w untuk mengosongkan sel
    Studi Literatur &
    \G & \G & \G & \w & \w & \w & \w & \w & \w & \w & \w & \w & \w & \w & \w & \w \\
    \hline

    Pengumpulan Dataset &
    \w & \w & \w & \G & \G & \G & \w & \w & \w & \w & \w & \w & \w & \w & \w & \w \\
    \hline

    Pembuatan Model &
    \w & \w & \w & \w & \w & \G & \G & \G & \G & \G & \G & \G & \G & \w & \w & \w \\
    \hline

    % Pengujian dan Analisa &
    % \w & \w & \w & \w & \w & \w & \w & \w & \w & \w & \w & \G & \G & \G & \w & \w \\
    % \hline
    
    Pembuatan Laporan &
    \G & \G & \G & \G & \G & \G & \G & \G & \G & \G & \G & \G & \G & \G & \G & \G \\
    \hline

  \end{tabular}
\end{table}